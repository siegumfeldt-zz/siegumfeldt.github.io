\documentclass[a4paper]{article} 
\usepackage[utf8]{inputenc}
\usepackage[danish]{babel}
\usepackage{currvita}
\usepackage{fullpage}
\usepackage{charter}


% sudo apt-get install texlive-latex-base 
% sudo apt-get install texlive-lang-danish 
% sudo apt-get install texlive-latex-extra
% sudo apt-get install texlive-fonts-recommended


\newcommand*{\ac}[1]{\mbox{#1}}
\cvplace{København}
\pagenumbering{gobble}

\begin{document} 

\setlength{\cvlabelwidth}{35mm}

\begin{cv}{Frederik Siegumfeldt\\{\large \itshape Curriculum Vitae}}

\begin{cvlist}{Erhvervserfaring}

  \item[2013-]       Datawarehouse-arkitekt, SDC

  \item[2007-2013]   Analytiker, Geomatic a/s
      
  \item[2005-2007]   Konsulent, Epinion A/S
      
  \item[2000-2005]   Diverse studentermedhjælperjobs (bl.a. 
                     C{}\verb!++!-programmering og brugersupport), \\ 
                     Institut for Statskundskab, Aarhus Universitet.

\end{cvlist}

\begin{cvlist}{Uddannelse}          

  \item \emph{Cand. scient. pol}, 2005 \\
    Aarhus Universitet (10,6 i snit) 
    
  \item \emph{Bac. scient. pol}, 2001 \\
    Aarhus Universitet (9,0 i snit) 
    
  \item \emph{Exam. art.} i filosofi, 1997 \\
    Aarhus Universitet

  \item \emph{Matematisk studentereksamen}, 1995 \\
    Aalborg Katedralskole (10,7 i snit)

\end{cvlist}

\begin{cvlist}{Profil}

  \item

%   `Fleksibel', `alsidig' og `løsningsorienteret' er nok de tre ord, der 
%   Fleksibel, alsidig og løsningsorienteret er nok de tre karakteristika,
    Fleksibilitet, alsidighed og og en løsningsorienteret tilgang til 
    opgaverne er nok de tre egenskaber, der bedst opsummerer mig rent 
    fagligt. I mine hidtidige jobs har min rolle altid været meget løst 
    afgrænset, og det har givet mig mulighed for at arbejde på mange 
    forskellige områder, hvilket jeg trives med.

    Jeg er hurtig til at sætte mig ind i nye problemstillinger, 
    danne mig et overblik over behov og muligheder og finde steder, 
    hvor jeg kan bidrage.

    Som person er jeg ganske udadvendt, og da jeg også spænder ret bredt
    fagligt har jeg haft nemt ved at danne mig netværk på mine 
    arbejdspladser. Det har givet mig gode muligheder for at fungere som
    fagligt bindeled mellem de forskellige kolleger, der har været 
    involveret i opgaveløsningen.

\end{cvlist}

\begin{cvlist}{Kompetencer}          

  \item[\emph{Softwareudviking}]

    % Store træk: Vise bredden
    Jeg har skrevet kode i arbejdsmæssig sammenhæng siden min 
    studietid, først i C{}\verb!++! og siden i SQL og Python 
    (med afstikkere til C\# og VBA). Desuden har jeg erfaring med en 
    række mere specialiserede sprog til statistisk modellering 
    (\ac{STATA}, \ac{SAS} og \ac{SPSS}).

    % Mere konkret: Hvad har jeg lavet?
    Min udviklingserfaring har primært drejet sig om design, 
    implementering og vedligeholdelse af dataintegrationsflows i SQL,
    suppleret med Python som generel ``schweizer\-kniv'' til
    sammenkobling af API'er og udvikling af små værktøjer (til
    f.eks. parsing af data og automatisering af modelbygning og
    rapportering). I mit nuværende job går der mindre tid med at 
    skrive kode, men mere tid med kravsspecs og reviews.

    % Ikke bare kode, også udviklingsproces/-redskaber
    Jeg har fire års erfaring med versionsstyring i Mercurial, og 
    medvirker for tiden til at integrere versionsstyring i udviklings-
    og release\-processerne for de rent SQL-baserede løsninger i 
    datavarehuset. I det hele taget har mit nuværende arbejde lært mig 
    en del om vigtigheden af de ting, der ligger før, efter og omkring 
    arbejdet med at skrive kode.

  \item[\emph{Data}]

    % Store træk: Det har handlet om data
    Mit hidtidige arbejde med at udvikle og vedligeholde software har 
    primært drejet sig om data: Om at designe datastrukturer til at 
    dække konkrete rappor\-terings- og analysebehov og derefter få data 
    til at flyde igennem fra kildesystem til færdig model og 
    afrapportering. Jeg har med årene oparbejdet meget solide 
    SQL-kompetencer (SQL Server og Teradata).

    % Udviklingen har været mod det mere arkitekturelle
    Med tiden har mit fokus flyttet sig fra analyse og afrapportering 
    mod databasedesign/datamodellering og arbejdet med den mere 
    arkitektoniske side af BI og datawarehousing. Jeg trives godt med
    at have en rolle der kombinerer rimelig lavpraktisk arbejde med kode
    (kravsspecifikationer, reviews og lidt ``rigtig'' udvikling) med 
    design af nye lag i datavarehuset og udarbejdelse af kodestandarder 
    og processer.

  \item[\emph{Statistisk modellering}]

    Som studerende og i mine første jobs arbejdede jeg meget med 
    forskellige typer af statistisk modellering. Jeg har modelleret churn 
    og livstidsværdi af kunder, lavet værdiansættelsesmodeller for 
    fast ejendom og meget andet.

    I mit nuværende arbejde har jeg ikke længere den slags opgaver,
    og selv om jeg heller ikke ser det som en stor del af mit arbejde 
    fremover, så skader gode statistikkundskaber aldrig -- særligt 
    ikke når man arbejder i finanssektoren.

  \item[\emph{Forretningsforståelse}]

    % Store træk: 
    I alle mine jobs har en væsentlig del af mit arbejde været at 
    sætte mig ind i de forretningsmæssige krav til projekter og omsætte 
    dem til en konkret løsning.

    Da jeg var i konsulentbranchen var forretningsområderne mange og 
    projekterne relativt små, og i mit nuværende arbejde er jeg kommet 
    mere i dybden med det finansielle område. Jeg har gode evner til 
    at tilegne mig ny viden -- både IT-fagligt og forretningsmæssigt.

  \item[\emph{Formidling}] 

    I mine jobs på Epinion og Geomatic var det en væsentlig del af 
    mit arbejde at præsentere analyseprojekter for kunder og formidle såvel 
    metodisk grundlag som konkrete konklusioner i skrift og tale.

    Formidling af den slags fylder ikke meget i mit nuværende job, men
    jeg lægger stadig meget stor vægt på at skrive klart og præcist, hvad end
    det er i en email eller en kravsspecifikation.

  \item[\emph{Sprog}]

    Dansk (Modersmål)

    Engelsk (Flydende, skrift og tale)

\end{cvlist}

\begin{cvlist}{Personlige data}

  \item[\emph{Alder}]       38 år (Født 28. oktober 1976)

  \item[\emph{Familie}]     Gift med Kirstine, far til Benjamin og Jonathan. 

  \item[\emph{Email}]       siegumfeldt@gmail.com

  \item[\emph{Telefon}]     51 48 10 28

  \item[\emph{Adresse}]     Oxford Have 227\\ 
                            2300 København S

\end{cvlist}
\end{cv}
\end{document}
